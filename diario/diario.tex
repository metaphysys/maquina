\documentclass[a4paper, 11pt, titlepage]{book}
\usepackage{fancyhdr}
\usepackage{graphicx}
\usepackage{imakeidx}
\usepackage{makeidx}
\usepackage{mathtools}
\usepackage[spanish]{babel}
\usepackage{eurosym}
\usepackage{hyperref}
\usepackage{amssymb}
\usepackage{listings}
\usepackage{xcolor}

\setcounter{secnumdepth}{5}
\setcounter{tocdepth}{5}

\title{Diario}
\author{Francisco Javier Balón Aguilar}

\begin{document}

\maketitle
\renewcommand{\contentsname}{Índice}
\tableofcontents
\newpage

\chapter{El Administrador}

    \section{Watchdog}

    El watchdog (perro guardían o vigilante) del kernel de Linux se utiliza para 
    monitorizar y vigilar un sistema que se está ejecutando, de forma que pueda 
    ser reiniciado automáticamente en caso de fallo de software irrecuperables.

    El módulo de watchdog es específico para el hardware o chip que se utiliza.

    Los usuarios, por norma general, no necesitan de vigilancia en sus sistemas, 
    pues pueden ellos mismos restablecer el sistema manualmente; así está pensado
    para dispositivos críticos que requieran de restablecimientos sin intervención
    humana, por ejemplo servidores en una ubicación remota o equipos integrados 
    en naves espaciales.

    En su intervención es necesario proceder con precaución, ya que configuraciones
    incorrectas o erróneas podrían provocar:

    \begin{itemize}
        \item Bucles infinitos de reinicios.
        \item Corrupción de archivos debido al hard reset.
        \item Reinicios aleatorios impredecibles.
    \end{itemize}

    Por lo que es preferible evitar el uso de servidores en vivo para probar el 
    funcionamiento de watchdog.

    \subsection{Watchdog module}

        La funcionalidad de watchdog sobre hardware configura un temporizador 
        que agota el tiempo de espera después de un período predeterminado.
        El software de watchdog actualiza periódicamente el temporizador del 
        hardware, y si éste deja de actualizarse luego del período establecido 
        el temporizador realizará un restablecimiento de hardware del dispositivo.
        Para que un temporizador watchdog sea funcional, el fabricante de la placa 
        base tiene que permitir esta funcionalidad en el chip.

        A menudo, la documentación del fabricante de hardware no deja claro 
        sobre si se implementó o no esta funcionalidad. En ese caso hay que 
        testearlo directamente.

        Además es necesario el módulo del kernel de vigilancia correcto, que será 
        cargado en el sistema Linux. Diferentes chips utilizan diferentes módulos,
        por ejemplo:

        \begin{itemize}
            \item Intel utiliza \textit{iTCO\_wdt}.
            \item El hardware de HP utiliza \textit{hpwdt}.
            \item Los mainframes de IBM utilizan \textit{vmwatchdog}.
            \item Xen VM utiuliza \textit{xen\_wdt}.
        \end{itemize}

        Una vez corroborado la carga del módulo es posible verificar \textit{/dev/watchdog}
        en el sistema Linux. Si el archivo está presente, significa que se cargó el controlador.
        El sistema escribe periódicamente a \textit{/dev/watchdog} (esta acción
        suele llamarse coloquialmente <<patear o alimentar al perro guardián>> / <<kicking
        or feeding the watchdog>>). Si el sistema falla y no patea o alimenta al 
        perro, el sistema se restablecerá completamente.

    \subsection{Watchdog daemon}

        El demonio de watchdog abre el dispositivo y proporciona la actualización 
        necesaria para evitar que el sistema se restablezca. Éste puede comprobar 
        el espacio de la tabla de procesos, el uso de memoria (memory usage), la 
        accesibilidad de archivos, la carga de trabajo, desbordamientos en la tabla de 
        archivos, ping a direcciones IP, tráfico de interfaz de red, temperatura, 
        procesos en ejecución, etc. Como ya hemos visto, si las pruebas fallan 
        el perro provocará el apagado.

    \subsection{Iniciando y parando el demonio de watchdog}

        El demonio de watchdog debería iniciarse en el momento del arranque. 
        Podemos verificar si está iniciado con:

        \begin{lstlisting}[language=bash]
    ps -af | grep watch*\end{lstlisting}

        Si el kernel no estuviera compilado con \textit{CONFIG\_WATCHDOG\_NOWAYOUT}
        y cerramos correctamente \textit{/dev/watchdog} no habrá reinicio. Podemos 
        escribir un carácter $V$ en \textit{/dev/watchdog} y luego cerrar el archivo;
        lo que detendrá al perro guardián.

        \subsubsection{Testeando watchdog} Si queremos comprobar que el control de 
        hardware está funcionando, podemos lanzar la siguiente orden con privilegios 
        root:

        \begin{lstlisting}[language=bash]
    cat >> /dev/watchdog\end{lstlisting}

        Una vez haya pasado el tiempo esperado, dependiendo de la configuración 
        del kernel, el sistema realizará el hard reboot.

    \subsection{Configuración del demonio de watchdog}

        Una vez conocemos watchdog, podemos iniciar su configuración. El fichero de 
        configuración de watchdog se encuentra en \textit{/etc/watchdog.conf} 
        (es posible que haya que instalar el paquete \textit{watchdog} en la distribución 
        utilizada).

        \begin{lstlisting}[language=bash]
    # cat /etc/watchdog.conf
    realtime = yes
    priority = 1\end{lstlisting}

\chapter{El Ingeniero}

\chapter{El Programador}

    \section{El programador C}

    \section{El programador C\#}

    \section{El programador Java}

    \section{El programador Python}

\chapter{El Hacker}

    \section{Reconocimiento}
            
        \subsection{nmap}

            % https://nmap.org/man/es/man-host-discovery.html

            \subsubsection{\textbf{-sL (Sondeo de lista)}} El sondeo de lista es un tipo de descubrimiento
            de sistemas que tan solo lista cada equipo de la/s red/es especificada/s, sin enviar paquetes 
            de ningún tipo a los objetivos. Por omisión, Nmap va a realizar una resolución inversa DNS en 
            los equipos, para obtener sus nombres. Es sorprendente cuanta información útil se puede obtener 
            del nombre de un sistema. Por ejemplo fw.chi.playboy.com es el cortafuegos de la oficina en 
            Chicago de Playboy Enterprises. Adicionalmente, al final, Nmap reporta el número total de 
            direcciones IP. El sondeo de lista es una buena forma de asegurarse de que tenemos las 
            direcciones IP correctas de nuestros objetivos. Si se encontraran nombres de dominio que no 
            reconoces, vale la pena investigar un poco más, para evitar realizar un análisis de la red 
            de la empresa equivocada.

            Ya que la idea es simplemente emitir un listado de los sistemas objetivo, las opciones de 
            mayor nivel de funcionalidad como análisis de puertos, detección de sistema operativo, o 
            análisis ping no pueden combinarse con este sondeo. Si desea deshabilitar el análisis ping 
            aún realizando dicha funcionalidad de mayor nivel, compruebe la documentación de la opción -P0.

            \subsubsection{\textbf{-sP (Sondeo ping)}} Esta opción le indica a Nmap que únicamente realice 
            descubrimiento de sistemas mediante un sondeo ping, y que luego emita un listado de los equipos 
            que respondieron al mismo. No se realizan más sondeos (como un análisis de puertos o detección 
            de sistema operativo). A diferencia del sondeo de lista, el análisis ping es intrusivo, ya que 
            envía paquetes a los objetivos, pero es usualmente utilizado con el mismo propósito. Permite 
            un reconocimiento liviano de la red objetivo sin llamar mucho la atención. El saber cuántos 
            equipos se encuentran activos es de mayor valor para los atacantes que el listado de cada una 
            de las IP y nombres proporcionado por el sondeo de lista.

            De la misma forma, los administradores de sistemas suelen encontrar valiosa esta opción. 
            Puede ser fácilmente utilizada para contabilizar las máquinas disponibles en una red, o 
            monitorizar servidores. A esto se lo suele llamar barrido ping, y es más fiable que hacer 
            ping a la dirección de broadcast, ya que algunos equipos no responden a ese tipo de consultas.

            La opción -sP envía una solicitud de eco ICMP y un paquete TCP al puerto 80 por omisión. 
            Cuando un usuario sin privilegios ejecuta Nmap se envía un paquete SYN (utilizando la llamada 
            connect()) al puerto 80 del objetivo. Cuando un usuario privilegiado intenta analizar objetivos 
            en la red Ethernet local se utilizan solicitudes ARP (-PR) a no ser que se especifique la 
            opción --send-ip.

            La opción -sP puede combinarse con cualquiera de las opciones de sondas de descubrimiento 
            (las opciones -P*, excepto -P0) para disponer de mayor flexibilidad. Si se utilizan cualquiera 
            de las opciones de sondas de descubrimiento y número de puerto, se ignoran las sondas por omisión 
            (ACK y solicitud de eco ICMP). Se recomienda utilizar estas técnicas si hay un cortafuegos con un 
            filtrado estricto entre el sistema que ejecuta Nmap y la red objetivo. Si no se hace así pueden 
            llegar a pasarse por alto ciertos equipos, ya que el cortafuegos anularía las sondas o las 
            respuestas a las mismas.

            \subsubsection{\textbf{-P0 (No ping)}} Con esta opción, Nmap no realiza la etapa de descubrimiento. 
            Bajo circunstancias normales, Nmap utiliza dicha etapa para determinar qué máquinas se encuentran 
            activas para hacer un análisis más agresivo. Por omisión, Nmap sólo realiza ese tipo de sondeos, 
            como análisis de puertos, detección de versión o de sistema operativo contra los equipos que se 
            están «vivos». Si se deshabilita el descubrimiento de sistemas con la opción -P0 entonces Nmap 
            utilizará las funciones de análisis solicitadas contra todas las direcciones IP especificadas. 
            Por lo tanto, si se especifica una red del tamaño de una clase B cuyo espacio de direccionamiento 
            es de 16 bits, en la línea de órdenes, se analizará cada una de las 65.536 direcciones IP. El 
            segundo carácter en la opción -P0 es un cero, y no la letra O. Al igual que con el sondeo de lista, 
            se evita el descubrimiento apropiado de sistemas, pero, en vez de detenerse y emitir un listado de 
            objetivos, Nmap continúa y realiza las funciones solicitadas como si cada IP objetivo se encontrara 
            activa.


\chapter{El sistema}

    \section{Servidor gráfico}

    \section{Gestor de ventanas}

        \subsection{awesomewm}

        \subsection{dwm}

    \section{Emulador de terminal}

        \subsection{urxvt}

    \section{Componentes físicos}

        \subsection{compton}

    \section{Editor}

        \subsection{emacs}

\end{document}